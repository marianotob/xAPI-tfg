\documentclass[a4paper,11pt,titlepage]{scrbook}
\usepackage[utf8]{inputenc}
\usepackage[spanish]{babel}

% \usepackage[style=list, number=none]{glossary} %si se va a usar glosario, quitar marca de comentario
%\usepackage{titlesec}
%\usepackage{palatino} %usar fot palatino en vez de times roman

%\decimalpoint %revisar
%\usepackage{dcolumn} %revisat
%\newcolumntype{.}{D{.}{\esperiod}{-1}}
%\makeatletter
%\addto\shorthandsspanish{\let\esperiod\es@period@code}
%\makeatother


%\usepackage[chapter]{algorithm}
%\RequirePackage{verbatim}
%\RequirePackage[Glenn]{fncychap}
\usepackage{fancyhdr}
\usepackage{graphicx}
\usepackage{afterpage}
\usepackage{longtable}
\usepackage{xcolor}
\definecolor{portada}{RGB}{239,206,53}
\definecolor{base}{RGB}{35,31,32}
\usepackage{pdfpages}


%Instrucciones para poder escribir código y mostrarlo de manera elegante:
\definecolor{gray97}{gray}{.97}
\definecolor{gray75}{gray}{.75}
\definecolor{gray45}{gray}{.45}
\definecolor{gray30}{gray}{.94}

\usepackage{listings}
\lstset{ frame=Ltb,
framerule=0pt,
aboveskip=0.5cm,
framextopmargin=3pt,
framexbottommargin=3pt,
framexleftmargin=0.4cm,
framesep=0pt,
rulesep=.4pt,
backgroundcolor=\color{gray97},
rulesepcolor=\color{black},
%
stringstyle=\ttfamily,
showstringspaces = false,
basicstyle=\small\ttfamily,
commentstyle=\color{gray45},
keywordstyle=\bfseries,
%
numbers=left,
numbersep=15pt,
numberstyle=\tiny,
numberfirstline = false,
breaklines=true,
literate={á}{{\'a}}1 {Á}{{\'A}}1 {é}{{\'e}}1 {É}{{\'e}}1 {í}{{\'i}}1  {Í}{{\'I}}1  {ó}{{\'o}}1  {Ó}{{\'O}}1  {ú}{{\'u}}1  {Ú}{{\'U}}1  {Ñ}{{\~N}}1 {ñ}{{\~n}}1 ,
}



% minimizar fragmentado de listados
\lstnewenvironment{listing}[1][]
   {\lstset{#1}\pagebreak[0]}{\pagebreak[0]}

\lstdefinestyle{Consola}
   {basicstyle=\scriptsize\bf\ttfamily,
    backgroundcolor=\color{gray30},
    frame=single,
    numbers=none
   }
\lstdefinestyle{C}
	{basicstyle=\scriptsize,
	frame=single,
	language=C,
	numbers=left
	}
\lstdefinestyle{CodigoC++}
        {basicstyle=\small,
	frame=single,
	backgroundcolor=\color{gray30},
	language=C++,
	numbers=left
 	}
\lstdefinestyle{PHP}
	{basicstyle=\scriptsize,
%        {basicstyle=\small,
	frame=single,
	language=PHP,
	numbers=left
	}
	



% ********************************************************************
% Información sobre el TFG. Comentar lo que NO se desee añadir y sustituir con la información correcta.
% ********************************************************************
\newcommand{\myTitle}{Título del Trabajo de Fin de Grado}
\newcommand{\mySubtitle}{Subtítulo del proyecto}
\newcommand{\myDegree}{Grado en Ingeniería Multimedia}
\newcommand{\myName}{Nombre Apellido1 Apellido2 (alumno)}
\newcommand{\myProf}{Nombre Apellido1 Apellido2 (tutor1)}
\newcommand{\myOtherProf}{Nombre Apellido1 Apellido2 (tutor2)}
\newcommand{\myFaculty}{Escuela Politécnica Superior de la Universidad de Alicante}
\newcommand{\myFacultyShort}{EPS UA}
\newcommand{\depTutorOne}{Departamento del tutor}
\newcommand{\depTutorTwo}{Departamento del cotutor}


\newcommand{\myUni}{\protect{Universidad de Alicante}}
\newcommand{\myLocation}{Alicante}
\newcommand{\myTime}{\today}
%\newcommand{\myVersion}{Version 0.1}

\newcommand{\logoGrado}{imagenes/logoim.jpg}
\newcommand{\logoFacultad}{imagenes/logoeps.jpg}
\newcommand{\logoUniversidad}{imagenes/logoua.jpg}

\usepackage{url}

% Definición de comandos que me son útiles:
%\renewcommand{\indexname}{Índice alfabético}
%\renewcommand{\glossaryname}{Glosario}

\pagestyle{fancy}
\fancyhf{}
\fancyhead[LO]{\leftmark}
\fancyhead[RE]{\rightmark}
\fancyhead[RO,LE]{\textbf{\thepage}}
\renewcommand{\chaptermark}[1]{\markboth{\textbf{#1}}{}}
\renewcommand{\sectionmark}[1]{\markright{\textbf{\thesection. #1}}}


\setlength{\headheight}{1.5\headheight}

\newcommand{\HRule}{\rule{\linewidth}{0.5mm}}
%Definimos los tipos teorema, ejemplo y definición podremos usar estos tipos
%simplemente poniendo \begin{teorema} \end{teorema} ...
\newtheorem{teorema}{Teorema}[chapter]
\newtheorem{ejemplo}{Ejemplo}[chapter]
\newtheorem{definicion}{Definición}[chapter]
 
\newcommand{\bigrule}{\titlerule[0.5mm]}


%Para conseguir que en las páginas en blanco no ponga cabeceras
\makeatletter
\def\clearpage{%
  \ifvmode
    \ifnum \@dbltopnum =\m@ne
      \ifdim \pagetotal <\topskip
        \hbox{}
      \fi
    \fi
  \fi
  \newpage
  \thispagestyle{empty}
  \write\m@ne{}
  \vbox{}
  \penalty -\@Mi
}
\makeatother

\usepackage[pdfborder={000}]{hyperref} %referencia
\hypersetup{
pdfauthor = {\myName (email (en) ua (punto) es)},
pdftitle = {\myTitle},
pdfsubject = {},
pdfkeywords = {palabra_clave1, palabra_clave2, palabra_clave3, ...},
pdfcreator = {LaTeX con el paquete ....},
pdfproducer = {pdflatex}
}
%AQUI COMIENZA LA LISTA DE FICHEROS A INCLUIR



\begin{document}
\renewcommand{\listtablename}{Índice de tablas} %para sustituir la palabra cuadro por tabla
\renewcommand{\tablename}{Tabla}
\renewcommand{\lstlistingname}{Listado}
\renewcommand{\lstlistlistingname}{Índice de \lstlistingname s}

\frontmatter
\input{portada/portada} %la portada en color
\input{portada/portada_2} %la portada en b/n

\chapter{Resumen}
Se debe incluir en la memoria un resumen de máximo 500 palabras, mejor se hace al final del trabajo con la visión global del mismo. Es obligatorio presentarlo para el tribunal. Se entrega en un fichero a parte y también lo podemos incluir en la memoria. Puede estar en las tres lenguas cas val y en


%%%%%%%%%%%%%%%%%%%%%%%Preliminares
\chapter*{Preámbulo}
\thispagestyle{empty}
Poner aquí un texto breve que debe incluir entre otras:
\begin{quote}
``las razones que han llevado a la realización del estudio, el tema, la finalidad y el alcance y también los agradecimientos por las ayudas, por ejemplo apoyo económico (becas y subvenciones) y las consultas y discusiones con los tutores y colegas de trabajo. \cite{UNE50136:97}''
\end{quote}

\cleardoublepage %salta a nueva página impar
% Aquí va la dedicatoria si la hubiese. Si no, comentar la(s) linea(s) siguientes
\chapter*{}
\setlength{\leftmargin}{0.5\textwidth}
\setlength{\parsep}{0cm}
\addtolength{\topsep}{0.5cm}
\begin{flushright}
\small\em{
A mi esposa Marganit, y a mis hijos Ella Rose y Daniel Adams,\\
sin los cuales habría podido acabar este libro dos años antes \footnote{Dedicatoria de Joseph J. Roman en ``An Introduction to Algebraic Topology''}
}
\end{flushright}


\cleardoublepage %salta a nueva página impar
% Aquí va la cita célebre si la hubiese. Si no, comentar la(s) linea(s) siguientes
\chapter*{}
\setlength{\leftmargin}{0.5\textwidth}
\setlength{\parsep}{0cm}
\addtolength{\topsep}{0.5cm}
\begin{flushright}
\small\em{
Si consigo ver más lejos\\
es porque he conseguido auparme\\ 
a hombros de gigantes
}
\end{flushright}
\begin{flushright}
\small{
Isaac Newton.
}
\end{flushright}
\cleardoublepage %salta a nueva página impar

%%%%%%%%%%%%%%%%%%%%%%%Fin preliminares

%\chapter*{Preámbulo}
\thispagestyle{empty}
Poner aquí un texto breve que debe incluir entre otras:
\begin{quote}
``las razones que han llevado a la realización del estudio, el tema, la finalidad y el alcance y también los agradecimientos por las ayudas, por ejemplo apoyo económico (becas y subvenciones) y las consultas y discusiones con los tutores y colegas de trabajo. \cite{UNE50136:97}''
\end{quote}

\cleardoublepage %salta a nueva página impar
% Aquí va la dedicatoria si la hubiese. Si no, comentar la(s) linea(s) siguientes
\chapter*{}
\setlength{\leftmargin}{0.5\textwidth}
\setlength{\parsep}{0cm}
\addtolength{\topsep}{0.5cm}
\begin{flushright}
\small\em{
A mi esposa Marganit, y a mis hijos Ella Rose y Daniel Adams,\\
sin los cuales habría podido acabar este libro dos años antes \footnote{Dedicatoria de Joseph J. Roman en ``An Introduction to Algebraic Topology''}
}
\end{flushright}


\cleardoublepage %salta a nueva página impar
% Aquí va la cita célebre si la hubiese. Si no, comentar la(s) linea(s) siguientes
\chapter*{}
\setlength{\leftmargin}{0.5\textwidth}
\setlength{\parsep}{0cm}
\addtolength{\topsep}{0.5cm}
\begin{flushright}
\small\em{
Si consigo ver más lejos\\
es porque he conseguido auparme\\ 
a hombros de gigantes
}
\end{flushright}
\begin{flushright}
\small{
Isaac Newton.
}
\end{flushright}
\cleardoublepage %salta a nueva página impar
 
%editar este texto (capitulos/preliminares.tex) para cambiar preámbulo, agradecimientos y dedicatorias
% En los preliminares podremos editar la siguiente información:
% [Preámbulo:] se describirán brevemente la motivación que ha originado la realización del TFG/TFM, así como de una breve descripción de los objetivos generales que se quieren alcanzar con el trabajo presentado.
% [Agradecimientos:] se podrá añadir las hojas necesarias para realizar los agradecimientos, a veces obligatorios, a las entidades y organismos colaboradores.
% [Dedicatoria:] se podrá añadir una única hoja con dedicatorias, su alineación será derecha y centradas de forma distribuida en la página.
% [Citas:] (frases célebres) se podrá añadir una única hoja con citas, su alineación será derecha y centradas de forma distribuida en la página.

%Con los siguientes comandos se genera el índice, indice de figuras, tablas y listados de código
\tableofcontents
\listoffigures
\listoftables
\lstlistoflistings

\mainmatter %entre frontmatter y mainmatter, la numeración es en romanos.

%a continuación se propone un esquema de trabajo que puede ser alterado justificadamente.

\chapter{Introducción}
En la introducción es donde se hará énfasis a la importancia de la temática, su vigencia y actualidad; se planteará el problema a investigar, así como el propósito o finalidad de la investigación. Debemos explicar el contexto donde se ubicará el trabajo.
Se pueden establecer secciones si así lo requiere la memoria.

\chapter{Modelo de negocio} %Estudio de viabilidad
\section{Lean Canvas}
Esta sección es opcional y depende de la tipología del trabajo a realizar. 

Es importante pensar y plasmar en el trabajo cómo nuestro proyecto va a aporta valor a los clientes, conocer sus necesidades o problemas y cómo les damos solución y cómo estos clientes van a pagar por ese valor que reciben. Además, es importante conocer el tipo de mercado en el que operamos y qué otros factores influyen en él.

Se pueden incluir análisis de DAFO y análisis de los riesgos


%Objetivos
\chapter{Objetivos} \label{objetivos}
\section{Generales}
El objetivo general del proyecto es crear ...

\section{Específicos}
\begin{enumerate}
\item objetivo 1
\item objetivo 2
\end{enumerate}


%Marco teórico o estado del arte o estudio del problema
\chapter{Marco Teórico} \label{marcoteorico}


\chapter{Metodología} \label{metodologia}

\chapter{Análisis y especificación} \label{analisis}
Análisis de los requerimientos funcionales y no funcionales según la norma IEEE 830
%Casos de uso
%
\chapter{Diseño} \label{disenyo}
\section{Arquitectura seleccionada}
\section{Tecnologías} %stack tecnológico
\section{Diagrama de clases}
\section{Mockups}
% Podemos incluir diseños, mockups, diseño de intertaces, prototipos, guías de estilo, user experience, diseño de pruebas y validación,...


\chapter{Implementación} \label{implementacion}
% Descripción del proceso, implementación y pruebas 
% Frontend, backend
% Despliegue en producción, pruebas y validación

\chapter{Resultados}\label{resultados}

\chapter{Conclusiones} 
\section{Conclusiones}\label{conclusiones}
\section{Líneas de trabajo futuras} \label{futuro}



%%\nocite{*} %incluye TODOS los documentos de la base de datos bibliográfica sean o no citados en el texto
\bibliography{bibliografia/bibliografia}\addcontentsline{toc}{chapter}{Bibliografía} %sustituir bibliografía con el nombre del fichero bibtex con la bibliografía
\bibliographystyle{apalike}
%
\appendix
\chapter{Anexo I. Esquema de base de datos}
Aquí vendría el anexo I 

\chapter{Anexo II. Otros...}
Aquí vendría el anexo II 

%\input{glosario/entradas_glosario}
% \addcontentsline{toc}{chapter}{Glosario} %si se usa glosario hay que añadirlo al índice
% \printglossary %muestra el glosario

\end{document}
