\chapter{Introducción}
\section{¡Importante!, leer primero}

Este texto está escrito pensando en orientar a los alumnos que usarán \LaTeX para escribir sus TFG de Ingeniería Multimedia.

Contiene información útil para aquellos que no tengan experiencia previa en \LaTeX así como algunos datos acerca de cómo escribir mejor su TFG.

A continuación, se ofrece una copia de la información que hay en el libro de estilo para la realización de los TFG de la EPS de la Universidad de Alicante.

En el siguiente capítulo (página \pageref{marcoteorico}) encontrarás algunos ejemplos de cómo hacer listas, tablas y otras estructuras de un texto en \LaTeX. Con un poco de paciencia, estudia cómo se hacen estas cosas y luego aplícalas en tus documentos.

\section{Estructura de un TFG}

En caso de que el TFG/TFM tenga como finalidad la elaboración de un proyecto o un 
informe científico o técnico, deberá ajustarse a lo dispuesto en las normas UNE 
157001:2002 y UNE 50135:1996 respectivamente.

Si el TFG/TFM tiene por finalidad la elaboración de un trabajo monográfico, el 
documento presentado deberá constar de las siguientes partes, teniendo como base la 
norma UNE 50136:1997.

\begin{description}
\item[Preámbulo:] se describirán brevemente la motivación que ha originado la realización del TFG/TFM, así como de una breve descripción de los objetivos generales que se quieren alcanzar con el trabajo presentado.
\item[Agradecimientos:] se podrá añadir las hojas necesarias para realizar los agradecimientos, a veces obligatorios, a las entidades y organismos colaboradores.
\item[Dedicatoria:] se podrá añadir una única hoja con dedicatorias, su alineación será derecha y centradas de forma distribuida en la página.
\item[Citas:] (frases célebres) se podrá añadir una única hoja con citas, su alineación será derecha y centradas de forma distribuida en la página.
\item[Índices:] cada índice debe comenzar en una nueva página, se incluirán los índices que se estimen necesarios (conforme UNE 50111:1989), en este orden:
\begin{description}
\item[Índice de contenidos:] (obligatorio siempre) se incluirá un índice de las secciones de las que se componga el documento, la numeración de las 
divisiones y subdivisiones utilizarán cifras arábigas (según UNE 50132:1994) y harán mención a la página del documento donde se ubiquen.
\item[Índice de figuras:] si el documento incluye figuras se podrá incluir también un índice con su relación, indicando la página donde se ubiquen.
\item[Índice de tablas:] en caso de existir en el texto, ídem que el anterior.
\item[Índice de abreviaturas, siglas, símbolos, etc.:] en caso de ser necesarios se podrá incluir cada uno de ellos.
\end{description}
\item[Cuerpo del documento:] en el contenido del documento se da flexibilidad para su organización y se puede estructurar en las secciones que se considere. En todo caso obligatoriamente se deberá, al menos, incluir los siguientes contenidos:
\begin{description}
\item[Introducción:] donde se hará énfasis a la importancia de la temática, su vigencia y actualidad; se planteará el problema a investigar, así como el propósito o finalidad de la investigación.
\item[Marco teórico o Estado del arte:] se hará mención a los elementos conceptuales que sirven de base para la investigación, estudios previos relacionados con el problema planteado, etc.
\item[Objetivos:] se establecerá el objetivo general y los específicos.
\item[Metodología:] se indicará el tipo o tipos de investigación, las técnicas y los procedimientos que serán utilizados para llevarla a cabo; se identificará la población y el tamaño de la muestra así como las técnicas e instrumentos de recolección de datos.
\item[Resultados:] incluirá los resultados de la investigación o trabajo, así como el análisis y la discusión de los mismos.
\end{description}
\item[Conclusiones:] obligatoriamente se incluirá una sección de conclusiones donde se realizará un resumen de los objetivos conseguidos así como de los resultados obtenidos si proceden.
\item[Bibliografía y referencias:] se incluirá también la relación de obras y materiales consultados y empleados en la elaboración de la memoria del TFG/TFM. La bibliografía y las referencias serán indexadas en orden alfabético (sistema nombre y fecha) o se numerará correlativamente según aparezca (sistema numérico). Se empleará la familia 1 como tipo de letra. Podrá utilizarse cualquier sistema bibliográfico normalizado predominante en la rama de conocimiento, estableciéndose como prioritarios el sistema ISO 690, sistema APA (American Psychological Association) o Harvard (no necesariamente en ese orden de preferencia). En esta plantilla Latex se propone usar el estilo APA indicándolo en la línea correspondiente como 
\begin{verbatim}
\bibliographystyle{apalike}
\end{verbatim}


\item[Anexos:] se podrá incluir los anexos que se consideren oportunos.

\end{description}



